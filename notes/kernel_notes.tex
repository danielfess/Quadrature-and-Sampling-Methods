\documentclass[english]{article}
\usepackage[T1]{fontenc}
\usepackage[latin9]{inputenc}
\usepackage{amsthm}
\usepackage{amsmath}
\usepackage{amssymb}
\usepackage{esint}

\makeatletter
%%%%%%%%%%%%%%%%%%%%%%%%%%%%%% Textclass specific LaTeX commands.
  \theoremstyle{definition}
  \newtheorem{defn}{\protect\definitionname}
  \theoremstyle{plain}
  \newtheorem{fact}{\protect\factname}
 \theoremstyle{definition}
  \newtheorem{example}{\protect\examplename}

\makeatother

\usepackage{babel}
  \providecommand{\definitionname}{Definition}
  \providecommand{\examplename}{Example}
  \providecommand{\factname}{Fact}

\begin{document}

\title{RKHS reading group: Notes}
\author{Oxford Kernels}
\maketitle

\section{Integral operator and covariance operator}
\begin{defn}
[Integral operator]\emph{ }Let $k$ be a continuous kernel on a
compact metric space $\mathcal{X}$, and let $\nu$ be a finite Borel
measure on $\mathcal{X}$. Let $S_{k}$ be the ``convolution'':
\begin{eqnarray*}
S_{k}\,:\,L_{2}(\mathcal{X};\nu) & \to & \mathcal{H}_{k},\\
\left(S_{k}f\right)\left(x\right) & = & \left\langle f,I_{k}k(x,\cdot)\right\rangle _{L_{2}(\mathcal{X};\nu)}\\
 & = & \int k(x,y)f(y)d\nu(y),\;\;f\in L_{2}(\mathcal{X};\nu),
\end{eqnarray*}
and $T_{k}=I_{k}S_{k}$ its composition with the inclusion $I_{k}\,:\,\mathcal{H}_{k}\hookrightarrow L_{2}(\mathcal{X};\nu)$.
$T_{k}$ is said to be the \emph{integral operator} of kernel $k$.
\end{defn}
To see that this definition is well posed, i.e. that $\text{im}(S_{k})\subset\mathcal{H}_{k}$,
use the same arguments as in the section on cross-covariance operators
below. 
\begin{fact}
$I_{k}=S_{k}^{*}$. In particular, $T_{k}=S_{k}^{*}S_{k}$ is self-adjoint.\end{fact}
\begin{proof}
Follows from
\begin{eqnarray*}
\left\langle h,S_{k}f\right\rangle _{\mathcal{H}_{k}} & = & \left\langle h,\int k(\cdot,y)f(y)d\nu(y)\right\rangle _{\mathcal{H}_{k}}\\
 & = & \int\left\langle h,k(\cdot,y)\right\rangle _{\mathcal{H}_{k}}f(y)d\nu(y)\\
 & = & \int h(y)f(y)d\nu(y)\\
 & = & \left\langle I_{k}h,f\right\rangle _{L_{2}(\mathcal{X};\nu)}.
\end{eqnarray*}
\end{proof}
\begin{defn}
[Covariance operator]$C_{\nu}=S_{k}S_{k}^{*}$ is the (uncentred)
covariance operator of $\nu$.\end{defn}
\begin{fact}
$\left\langle h,C_{\nu}g\right\rangle _{\mathcal{H}_{k}}=\left\langle h,S_{k}S_{k}^{*}g\right\rangle _{\mathcal{H}_{k}}=\left\langle I_{k}h,I_{k}g\right\rangle _{\mathcal{H}_{k}}=\int h(x)g(x)d\nu(x)$.
\end{fact}
Even though $C_{\nu}$ and $T_{k}$ differ only in the order of taking
inclusions (which seems irrelevant at first sight), they can be different
objects in general!
\begin{example}
If $\nu$ is a probability measure and $\mathcal{X}\subset\mathbb{R}^{p}$,
this is just like a (linear, uncentred) covariance matrix $C_{\nu}=\mathbb{E}_{X\sim\nu}\left[XX^{\top}\right]$
of a random vector $X$. If we consider covariance between scalar
projections of $X$, then 
\begin{eqnarray*}
\mathbb{E}\left[\left(a^{\top}X\right)\left(b^{\top}X\right)\right] & = & a^{\top}C_{\nu}b.
\end{eqnarray*}
Indeed, for a linear kernel, $\mathcal{H}_{k}$ is the set of linear
functionals on $\mathcal{X}$ and can be identified with $\mathbb{R}^{p}$
and $C_{\nu}:\mathbb{R}^{p}\to\mathbb{R}^{p}$. $T_{k}$ is a different
object though: to every $f\in L_{2}(\mathcal{X};\nu)$, it associates
a linear functional $x\mapsto\left(\int yf(y)d\nu(y)\right)^{\top}x$.
Notice that eigenfunctions of $T_{k}$ in this case must be linear,
so $f(y)=a^{\top}y$ for some $a\in\mathbb{R}^{p}$, so eigenvalue
equation reads
\begin{eqnarray*}
\int ya^{\top}yd\nu(y) & = & \lambda a\quad\Leftrightarrow\\
C_{\nu}a & = & \lambda a.
\end{eqnarray*}
Thus, even though $T_{k}$ is nominally an operator on $L_{2}$ it
has at most $p$ non-zero eigenvalues (which are the same as the eigenvalues
of $C_{\nu}$).
\end{example}

\section{Cross-covariance operator}

Let's take $\mathcal{F}=L^{2}(\mathcal{X}\times\mathcal{Y},P_{XY})$,
for some joint distribution $P_{XY}$ on $\mathcal{X}\times\mathcal{Y}$.
Further, assume that $l_{y}\doteq l(y,\cdot)\in\mathcal{F}$, and
$k_{x}\doteq k(x,\cdot)\in\mathcal{F}$, i.e., that:
\begin{eqnarray*}
\int l^{2}(y,\tilde{y})dP_{XY}(\tilde{x},\tilde{y}) & = & \int l^{2}(y,\tilde{y})dP_{Y}(\tilde{y})\\
 & < & \infty,
\end{eqnarray*}


\begin{eqnarray*}
\int k^{2}(x,\tilde{x})dP_{XY}(\tilde{x},\tilde{y}) & = & \int k^{2}(x,\tilde{x})dP_{X}(\tilde{x})\\
 & < & \infty.
\end{eqnarray*}

\begin{itemize}
\item Denote by$\iota_{k}:\mathcal{H}_{k}\to L^{2}(\mathcal{X}\times\mathcal{Y},P_{XY})$
and $\iota_{l}:\mathcal{H}_{l}\to L^{2}(\mathcal{X}\times\mathcal{Y},P_{XY})$
inclusions of the respective RKHSs into $\mathcal{F}$. Consider $R_{l}\;:\;L^{2}(\mathcal{X}\times\mathcal{Y},P_{XY})\to\mathbb{R}^{\mathcal{Y}}$,
given by
\begin{eqnarray*}
(R_{l}f)(y) & = & \left\langle f,\iota_{l}l_{y}\right\rangle _{\mathcal{F}}\\
 & = & \int f(\tilde{x},\tilde{y})l(\tilde{y},y)dP_{XY}(\tilde{x},\tilde{y}),
\end{eqnarray*}
and similarly for $R_{k}$. 
\item Let us show that $\textrm{im}(R_{l})$ is inside $\mathcal{H}_{l}$
(similarly, $\textrm{im}(R_{k})$ is inside $\mathcal{H}_{k}$). First,
function $(x,y)\mapsto f(x,y)l_{y}$ is Bochner $P_{XY}$-integrable
on $\mathcal{H}_{l}$, since:
\begin{eqnarray*}
\int\left\Vert f(x,y)l_{y}\right\Vert _{\mathcal{H}_{l}}dP_{XY}(x,y) & = & \int f(x,y)\sqrt{l(y,y)}dP_{XY}(x,y)\\
 & \leq & \left\Vert f\right\Vert _{L^{2}}\left[\int l(y,y)dP_{Y}(y)\right]^{1/2}\\
 & < & \infty.
\end{eqnarray*}
Second, the evaluation functional $\delta_{y'}:\mathcal{H}_{l}\to\mathbb{R}$
is bounded, so it commutes with the Bochner integral, i.e., 
\begin{eqnarray*}
(R_{l}f)(y') & = & \int\delta_{y'}\left[f(x,y)l_{y}\right]dP_{XY}(x,y)\\
 & = & \delta_{y'}\left[\int f(y)l_{y}dP_{XY}(x,y)\right],
\end{eqnarray*}
so that $R_{l}f=\int f(x,y)l_{y}dP_{XY}(x,y)\in\mathcal{H}_{l}$.
\item Let us show that $R_{l}^{*}$ is the inclusion $\iota_{l}$. For all
$h\in\mathcal{H}_{l}$, $u\mapsto\left\langle h,u\right\rangle _{\mathcal{H}_{l}}$
is a bounded linear operator, and thus: 
\begin{eqnarray*}
\left\langle h,R_{l}f\right\rangle _{\mathcal{H}_{l}} & = & \left\langle h,\int f(x,y)l_{y}dP_{XY}(x,y)\right\rangle _{\mathcal{H}_{l}}\\
 & = & \int f(x,y)\left\langle h,l_{y}\right\rangle _{\mathcal{H}_{k}}dP_{XY}(x,y)\\
 & = & \int f(x,y)h(y)dP(y)\\
 & = & \left\langle \iota_{l}h,f\right\rangle _{\mathcal{F}}.
\end{eqnarray*}

\item Thereby, we get the cross-covariance operators $C_{YX}=R_{l}R_{k}^{*}\,:\,\mathcal{H}_{k}\to\mathcal{H}_{l},$
and $C_{XY}=R_{k}R_{l}^{*}\,:\,\mathcal{H}_{l}\to\mathcal{H}_{k}$.
By construction, $C_{YX}=C_{XY}^{*}$. Now,
\end{itemize}
\begin{eqnarray*}
\left\langle g,C_{YX}f\right\rangle _{\mathcal{H}_{l}} & = & \left\langle g,R_{l}\iota_{X}f\right\rangle _{\mathcal{H}_{l}}\\
 & = & \left\langle \iota_{Y}g,\iota_{X}f\right\rangle _{\mathcal{F}}\\
 & = & \int f(x)g(y)dP_{XY}(x,y)\\
 & = & \mathbb{E}_{X,Y}\left[f(X)g(Y)\right].
\end{eqnarray*}

\end{document}
